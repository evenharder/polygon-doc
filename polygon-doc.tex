\documentclass{article}

\usepackage[hangul]{kotex}
\usepackage[hidelinks,unicode,bookmarks=true]{hyperref}
\hypersetup{bookmarksdepth=3}
\usepackage{fancyvrb}
\usepackage{color}
\usepackage{graphicx}
\usepackage{amsmath}
\usepackage{amsfonts}
\usepackage{amsthm}
\usepackage{textcomp}
\usepackage[top=3cm, left=3cm, right=3cm, bottom=2cm]{geometry}
\usepackage{lipsum}
\usepackage{titling}
\usepackage{colortbl}
\usepackage{standalone}

\setlength{\parindent}{0pt}
\setlength{\droptitle}{-3cm}
\setcounter{tocdepth}{1}

\title{간단하게 배워보는 Polygon}
\author{작성자 : \texttt{@evenharder}}
\date{최종날짜 : \today}
\begin{document}
    \maketitle
    이 문서는 Polygon (\url{https://polygon.codeforces.com/}) 의 간략한 소개를 다룹니다.
    \tableofcontents
    \newpage
    \section{Introduction}
    \documentclass{article}

\usepackage[hangul]{kotex}
\usepackage[unicode,bookmarks=true]{hyperref}
\usepackage{fancyvrb}
\usepackage{color}
\usepackage{graphicx}
\usepackage{amsmath}
\usepackage{amsfonts}
\usepackage[top=3cm, left=3cm, right=3cm, bottom=2cm]{geometry}
\usepackage{lipsum}
\begin{document}
    \begin{quotation}
        The mission of Polygon is to provide platform for creation of programming contest problems. Polygon supports the whole development cycle: 
        problem statement writing,
        test data preparing (generators supported),
        model solutions (including correct and wittingly incorrect),
        judging, and
        automatic validation.
        \begin{flushright}
            \textit{Polygon Index Page}
        \end{flushright}
    \end{quotation}
    메인 화면에 적혀있는 Polygon의 목표로, 프로그래밍 대회 문제 제작을 위한 플랫폼을 제공하는 것입니다.
\end{document}

    \section{Registration}
    \documentclass{article}

\usepackage[hangul]{kotex}
\usepackage[unicode,bookmarks=true]{hyperref}
\usepackage{fancyvrb}
\usepackage{color}
\usepackage{graphicx}
\usepackage{amsmath}
\usepackage{amsfonts}
\usepackage[top=3cm, left=3cm, right=3cm, bottom=2cm]{geometry}
\usepackage{lipsum}
\begin{document}
    메인 화면 우측 상단의 \textit{Register}로 등록 가능합니다.
\end{document}

    \section{Creating Problem}
    \documentclass{article}

\usepackage[hangul]{kotex}
\usepackage[unicode,bookmarks=true]{hyperref}
\usepackage{fancyvrb}
\usepackage{color}
\usepackage{graphicx}
\usepackage{amsmath}
\usepackage{amsfonts}
\usepackage[top=3cm, left=3cm, right=3cm, bottom=2cm]{geometry}
\usepackage{lipsum}
\begin{document}
    로그인을 하고, 상단의 \textit{New Problem}을 통해 문제의 코드네임(예시 : \verb|books|, \verb|hanoi|, \verb|holiday|)을 설정할 수 있습니다.
    \subsection*{Commit Changes}
    Git의 commit 기능처럼, Polygon 서버에 지금까지의 과정을 저장하고 싶을 때 우측 하단의 \textit{Commit Changes}를 누르면 Commit 과정을 진행할 수 있습니다.
\end{document}

    \section{General Info}
    \documentclass{article}

\usepackage[hangul]{kotex}
\usepackage[unicode,bookmarks=true]{hyperref}
\usepackage{fancyvrb}
\usepackage{color}
\usepackage{graphicx}
\usepackage{amsmath}
\usepackage{amsfonts}
\usepackage[top=3cm, left=3cm, right=3cm, bottom=2cm]{geometry}
\usepackage{lipsum}
\begin{document}
    General Info는 문제 자체의 특징을 다룹니다.
    \subsection{General Info의 구성}
    \begin{itemize}
        \item Input file : 입력 방식으로, \verb|stdin|을 권장합니다.
        \item Output file : 출력 방식으로, \verb|stdout|을 권장합니다.
        \item Time limit : 시간 제한(250ms - 15000ms)을 설정합니다.
        \item Memory limit : 메모리 제한(4MB - 1024MB)을 설정합니다.
        \item Interactive : 문제가 interactive한지의 여부를 의미합니다.
        \item Well-formed : 체크하면 whitespace나 control character를 Polygon이 알아서 처리해줍니다.
        \begin{itemize}
            \item 이 경우 자동으로 모든 test case가 whitespace 관련 제약조건을 만족하게 변경됩니다.
        \end{itemize}
    \end{itemize}
\end{document}

    \section{Statement}
    \documentclass{article}

\usepackage[hangul]{kotex}
\usepackage[unicode,bookmarks=true]{hyperref}
\usepackage{fancyvrb}
\usepackage{color}
\usepackage{graphicx}
\usepackage{amsmath}
\usepackage{amsfonts}
\usepackage[top=3cm, left=3cm, right=3cm, bottom=2cm]{geometry}
\usepackage{lipsum}
\begin{document}
    Polygon에서 문제 설명은 영어나 러시아어로 쓰는 게 권장됩니다.
    \subsection{Statement의 구성}
    \begin{itemize}
        \item Encoding : UTF-8 그대로가 권장됩니다.
        \item Name : 문제의 제목입니다.
        \item Legend : 배경 지식 및 문제 상황 설명입니다.
        \item Input Format : 입력 양식입니다.
        \item Output Format : 출력 양식입니다.
        \item Note : 예제가 나오는 이유 등을 적는 곳입니다.
        \item Tutorial : 문제 페이지에 보이진 않지만, 풀이를 적는 곳입니다.
    \end{itemize}
    여기까지의 내용은 페이지 하단의 Save를 눌러야 저장됩니다.
    \subsection{Statement 속 TeX 사용}
    Polygon의 Statement는 TeX을 이용해 컴파일됩니다.
    \subsubsection{TeX이 친숙하지 않을 때}
    \begin{itemize}
        \item TeX에선 연속된 whitespace는 하나로 처리됩니다.
        \item 줄바꿈은 \verb|\\|입니다.
        \item 수식은 \verb|$...$|나 \verb|$$...$$|의 꼴로 이루어집니다.
        \item 수식에서 윗첨자는 \verb|^|, 아랫첨자는 \verb|_|입니다.
        \item 그림도 집어넣을 수 있지만, Polygon 특성상 까다롭습니다.
        \item 특수 기호 사용법은 {\color{blue}\href{http://detexify.kirelabs.org/classify.html}{이 사이트}}에서 검색할 수 있습니다.
    \end{itemize}
    \subsubsection{TeX이 친숙할 때}
    \begin{itemize}
        \item 각 항목을 편집할 때 \verb|\usepackage| 구문은 필요없습니다.
        \item 복잡하거나 특수한 TeX 구문은 컴파일되지 않을 수 있습니다.
    \end{itemize}
\end{document}

    \section{Files}
    \documentclass{article}

\usepackage[hangul]{kotex}
\usepackage[unicode,bookmarks=true]{hyperref}
\usepackage{fancyvrb}
\usepackage{color}
\usepackage{graphicx}
\usepackage{amsmath}
\usepackage{amsfonts}
\usepackage[top=3cm, left=3cm, right=3cm, bottom=2cm]{geometry}
\usepackage{lipsum}
\begin{document}
    File은 풀이 파일을 제외한 모든 파일들이 보관되는 곳입니다.
    \begin{itemize}
        \item Resource Files : Polygon의 기본 파일과 추가적인 헤더 파일
        \item Source Files : Checker, Validator 등의 파일
        \item Attachment Files : 첨부사진 등 기타 파일
    \end{itemize}
\end{document}

    \section{testlib.h}
    \documentclass{article}

\usepackage[hangul]{kotex}
\usepackage[unicode,bookmarks=true]{hyperref}
\usepackage{fancyvrb}
\usepackage{color}
\usepackage{graphicx}
\usepackage{amsmath}
\usepackage{amsfonts}
\usepackage[top=3cm, left=3cm, right=3cm, bottom=2cm]{geometry}
\usepackage{lipsum}
\begin{document}
    \verb|testlib.h|는 Polygon에서 사용되는 채점용 C++ 헤더입니다. {\color{blue}\href{https://github.com/MikeMirzayanov/testlib}{이 사이트}}에서 다운받을 수 있으며, Polygon의 Checker와 Validator를 짜기 위해서는 필수적으로 알아야 합니다.
\end{document}

    \section{Tests}
    \documentclass{article}

\usepackage[hangul]{kotex}
\usepackage[unicode,bookmarks=true]{hyperref}
\usepackage{fancyvrb}
\usepackage{color}
\usepackage{graphicx}
\usepackage{amsmath}
\usepackage{amsfonts}
\usepackage[top=3cm, left=3cm, right=3cm, bottom=2cm]{geometry}
\usepackage{lipsum}
\begin{document}
    Test에는 test case들이 저장됩니다.
    \subsection{Test 만들기}
    \subsubsection{직접 추가}
    외부에서 만든 test case 추가는 Add Tests를 통해서 진행됩니다. 
    \begin{itemize}
        \item 직접 추가하기 : Data에 Ctrl CV하면 됩니다.
        \item From the archive : test case만 있는 zip 파일을 제출하면 됩니다.
        \item From the files : File을 올리면 됩니다.
    \end{itemize}
    다만 몇 개의 예제를 제외하고는 manual한 방법은 권장되지 않습니다.
    \subsubsection{Generator 작성 (Polygon)}
    \verb|testlib.h|를 사용해서 Generator를 만들 수 있습니다. 
    \begin{itemize}
        \item \verb|void registerGen(int argc, char* argv[], int randomGeneratorVersion)|\\
        이 파일이 Generator라는 것을 선언하는 함수입니다. \verb|argc|와 \verb|argv|는 main의 그것과 같습니다.
        \item \verb|double next()|\\
        0 이상 1 미만 무작위 실수를 반환합니다.
        \item \verb|T next(T n)|\\
        0 이상 \verb|n| 미만의 \verb|T|형 값을 반환합니다.
        \item \verb|T next(T from, T to)|\\
        \verb|from| 이상 \verb|to| 이하의 \verb|T|형 값을 반환합니다.
        \item \verb|std::string next(std::string ptm)|\\
        \verb|ptm|에 있는 Regex 기반 패턴에 맞춘 무작위 문자열을 반환합니다.
        \item \verb|T wnext(T n, int type)|\\
        0 이상 \verb|n| 미만의 T형 수 중, \verb|type|이 양수일 때는 큰 쪽으로, 음수일 때는 작은 쪽으로 가중치가 부여됩니다.
        \item \verb|void shuffle(_RandomAccessIter __first, _RandomAccessIter __last)|\\
        \verb|algorithm| 헤더의 \verb|std::random_shuffle| 대신 사용해야 하는 함수입니다.
        \item \verb|typename Container::value_type any(const Container& c)|\\
        \verb|c| 안에 있는 원소 중 하나가 무작위로 반환됩니다.
    \end{itemize}
    유의해야 할 점은 다음과 같습니다.
    \begin{itemize}
        \item[-] 모든 난수 생성 함수는 \verb|testlib.h|에 전역으로 선언되어 있는 \verb|rnd|을 이용해 호출해야 합니다.
        \item[-] \verb|rand()|나 \verb|srand()|는 사용할 수 없습니다(Polygon이 내부적으로 seeding을 진행합니다)
        \item[-] 인자는 \verb|stdin|대신 \verb|argc|와 \verb|argv|로 전달받습니다.
        \item[-] \verb|testlib.h|의 \verb|generator| 폴더 안에 예시가 동봉되어 있습니다.
    \end{itemize}
    이렇게 만든 Generator는 Files를 통해 업로드 가능하며, test를 생성할 때 Manual 옵션 대신 Script 옵션을 통해 넣을 수도 있습니다.
    \subsection{Generator 사용 (FreeMarker)}
    FreeMarker 기반의 스크립트를 이용하면 여러 개의 test를 Generator를 통해 생성할 수 있습니다. 다음은 예시입니다.
    \begin{verbatim}
    <#assign i = 1/>
    <#list 1..100 as i>
    randomGen 50000 50000 ${i} > $
    </#list>\end{verbatim}
    이 스크립트의 원리는 다음과 같습니다.
    \begin{itemize}
        \item 변수 \verb|i|를 첫 줄에서 할당합니다
        \item 둘째 줄에서 for loop을 돌립니다.
        \item \verb|randomGen|이라는 이름을 가진 generator에 인자 \verb|50000 50000 i| 를 넣어, 그 출력을 아직 만들어지지 않은 가장 낮은 번호의 test(\verb|$|)에 저장합니다.
        \begin{itemize}
            \item \verb|50000 50000|은 (이 경우) generator에게 전달하는 변수입니다.
            \item \verb|${i}|를 넣지 않으면 Polygon이 중복된 test case라고 판단해, 실행이 되지 않습니다.
            \item  또, 추가적으로 들어가는 \verb|${i}|는 seed의 역할도 합니다.
        \end{itemize}
    \end{itemize}
    for문은 중첩될 수도 있으며, \verb|<#assign num = num + 1/>| 등의 문구로 변수의 값을 조절할 수 있습니다.
\end{document}

\end{document}
