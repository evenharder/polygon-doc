\documentclass{article}

\usepackage[hangul]{kotex}
\usepackage[hidelinks,unicode,bookmarks=true]{hyperref}
\hypersetup{bookmarksdepth=3}
\usepackage{fancyvrb}
\usepackage{color}
\usepackage{graphicx}
\usepackage{amsmath}
\usepackage{amsfonts}
\usepackage{amsthm}
\usepackage{textcomp}
\usepackage[top=3cm, left=3cm, right=3cm, bottom=2cm]{geometry}
\usepackage{lipsum}
\usepackage{titling}
\usepackage{colortbl}
\usepackage{standalone}

\setlength{\parindent}{0pt}
\setlength{\droptitle}{-3cm}
\setcounter{tocdepth}{1}

\title{간단하게 배워보는 Polygon}
\author{작성자 : \texttt{@evenharder}}
\date{최종날짜 : \today}
\begin{document}
    \maketitle
    이 문서는 Polygon (\url{https://polygon.codeforces.com/}) 의 간략한 소개를 다룹니다.
    \tableofcontents
    \newpage
    \section{Introduction}
    \documentclass{article}

\usepackage[hangul]{kotex}
\usepackage[unicode,bookmarks=true]{hyperref}
\usepackage{fancyvrb}
\usepackage{color}
\usepackage{graphicx}
\usepackage{amsmath}
\usepackage{amsfonts}
\usepackage[top=3cm, left=3cm, right=3cm, bottom=2cm]{geometry}
\usepackage{lipsum}
\begin{document}
    \begin{quotation}
        The mission of Polygon is to provide platform for creation of programming contest problems. Polygon supports the whole development cycle: 
        problem statement writing,
        test data preparing (generators supported),
        model solutions (including correct and wittingly incorrect),
        judging, and
        automatic validation.
        \begin{flushright}
            \textit{Polygon Index Page}
        \end{flushright}
    \end{quotation}
    메인 화면에 적혀있는 Polygon의 목표로, 프로그래밍 대회 문제 제작을 위한 플랫폼을 제공하는 것입니다.
\end{document}

\end{document}
