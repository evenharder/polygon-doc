\documentclass{article}

\usepackage[hangul]{kotex}
\usepackage[hidelinks,unicode,bookmarks=true]{hyperref}
\hypersetup{bookmarksdepth=3}
\usepackage{fancyvrb}
\usepackage{color}
\usepackage{graphicx}
\usepackage{amsmath}
\usepackage{amsfonts}
\usepackage{amsthm}
\usepackage{textcomp}
\usepackage[top=3cm, left=3cm, right=3cm, bottom=2cm]{geometry}
\usepackage{lipsum}
\usepackage{titling}
\usepackage{colortbl}
\usepackage{standalone}

\setlength{\parindent}{0pt}
\setlength{\droptitle}{-3cm}
\setcounter{tocdepth}{1}

\title{간단하게 배워보는 Polygon}
\author{작성자 : \texttt{@evenharder}}
\date{최종날짜 : \today}
\begin{document}
    \maketitle
    이 문서는 Polygon (\url{https://polygon.codeforces.com/}) 의 간략한 소개를 다룹니다.
    \tableofcontents
    \newpage
    \section{Introduction}
    \documentclass{article}

\usepackage[hangul]{kotex}
\usepackage[unicode,bookmarks=true]{hyperref}
\usepackage{fancyvrb}
\usepackage{color}
\usepackage{graphicx}
\usepackage{amsmath}
\usepackage{amsfonts}
\usepackage[top=3cm, left=3cm, right=3cm, bottom=2cm]{geometry}
\usepackage{lipsum}
\begin{document}
    \begin{quotation}
        The mission of Polygon is to provide platform for creation of programming contest problems. Polygon supports the whole development cycle: 
        problem statement writing,
        test data preparing (generators supported),
        model solutions (including correct and wittingly incorrect),
        judging, and
        automatic validation.
        \begin{flushright}
            \textit{Polygon Index Page}
        \end{flushright}
    \end{quotation}
    메인 화면에 적혀있는 Polygon의 목표로, 프로그래밍 대회 문제 제작을 위한 플랫폼을 제공하는 것입니다.
\end{document}

    \section{Registration}
    \documentclass{article}

\usepackage[hangul]{kotex}
\usepackage[unicode,bookmarks=true]{hyperref}
\usepackage{fancyvrb}
\usepackage{color}
\usepackage{graphicx}
\usepackage{amsmath}
\usepackage{amsfonts}
\usepackage[top=3cm, left=3cm, right=3cm, bottom=2cm]{geometry}
\usepackage{lipsum}
\begin{document}
    메인 화면 우측 상단의 \textit{Register}로 등록 가능합니다.
\end{document}

    \section{Creating Problem}
    \documentclass{article}

\usepackage[hangul]{kotex}
\usepackage[unicode,bookmarks=true]{hyperref}
\usepackage{fancyvrb}
\usepackage{color}
\usepackage{graphicx}
\usepackage{amsmath}
\usepackage{amsfonts}
\usepackage[top=3cm, left=3cm, right=3cm, bottom=2cm]{geometry}
\usepackage{lipsum}
\begin{document}
    로그인을 하고, 상단의 \textit{New Problem}을 통해 문제의 코드네임(예시 : \verb|books|, \verb|hanoi|, \verb|holiday|)을 설정할 수 있습니다.
    \subsection*{Commit Changes}
    Git의 commit 기능처럼, Polygon 서버에 지금까지의 과정을 저장하고 싶을 때 우측 하단의 \textit{Commit Changes}를 누르면 Commit 과정을 진행할 수 있습니다.
\end{document}

    \section{General Info}
    \documentclass{article}

\usepackage[hangul]{kotex}
\usepackage[unicode,bookmarks=true]{hyperref}
\usepackage{fancyvrb}
\usepackage{color}
\usepackage{graphicx}
\usepackage{amsmath}
\usepackage{amsfonts}
\usepackage[top=3cm, left=3cm, right=3cm, bottom=2cm]{geometry}
\usepackage{lipsum}
\begin{document}
    General Info는 문제 자체의 특징을 다룹니다.
    \subsection{General Info의 구성}
    \begin{itemize}
        \item Input file : 입력 방식으로, \verb|stdin|을 권장합니다.
        \item Output file : 출력 방식으로, \verb|stdout|을 권장합니다.
        \item Time limit : 시간 제한(250ms - 15000ms)을 설정합니다.
        \item Memory limit : 메모리 제한(4MB - 1024MB)을 설정합니다.
        \item Interactive : 문제가 interactive한지의 여부를 의미합니다.
        \item Well-formed : 체크하면 whitespace나 control character를 Polygon이 알아서 처리해줍니다.
        \begin{itemize}
            \item 이 경우 자동으로 모든 test case가 whitespace 관련 제약조건을 만족하게 변경됩니다.
        \end{itemize}
    \end{itemize}
\end{document}

    \section{Statement}
    \documentclass{article}

\usepackage[hangul]{kotex}
\usepackage[unicode,bookmarks=true]{hyperref}
\usepackage{fancyvrb}
\usepackage{color}
\usepackage{graphicx}
\usepackage{amsmath}
\usepackage{amsfonts}
\usepackage[top=3cm, left=3cm, right=3cm, bottom=2cm]{geometry}
\usepackage{lipsum}
\begin{document}
    Polygon에서 문제 설명은 영어나 러시아어로 쓰는 게 권장됩니다.
    \subsection{Statement의 구성}
    \begin{itemize}
        \item Encoding : UTF-8 그대로가 권장됩니다.
        \item Name : 문제의 제목입니다.
        \item Legend : 배경 지식 및 문제 상황 설명입니다.
        \item Input Format : 입력 양식입니다.
        \item Output Format : 출력 양식입니다.
        \item Note : 예제가 나오는 이유 등을 적는 곳입니다.
        \item Tutorial : 문제 페이지에 보이진 않지만, 풀이를 적는 곳입니다.
    \end{itemize}
    여기까지의 내용은 페이지 하단의 Save를 눌러야 저장됩니다.
    \subsection{Statement 속 TeX 사용}
    Polygon의 Statement는 TeX을 이용해 컴파일됩니다.
    \subsubsection{TeX이 친숙하지 않을 때}
    \begin{itemize}
        \item TeX에선 연속된 whitespace는 하나로 처리됩니다.
        \item 줄바꿈은 \verb|\\|입니다.
        \item 수식은 \verb|$...$|나 \verb|$$...$$|의 꼴로 이루어집니다.
        \item 수식에서 윗첨자는 \verb|^|, 아랫첨자는 \verb|_|입니다.
        \item 그림도 집어넣을 수 있지만, Polygon 특성상 까다롭습니다.
        \item 특수 기호 사용법은 {\color{blue}\href{http://detexify.kirelabs.org/classify.html}{이 사이트}}에서 검색할 수 있습니다.
    \end{itemize}
    \subsubsection{TeX이 친숙할 때}
    \begin{itemize}
        \item 각 항목을 편집할 때 \verb|\usepackage| 구문은 필요없습니다.
        \item 복잡하거나 특수한 TeX 구문은 컴파일되지 않을 수 있습니다.
    \end{itemize}
\end{document}

    \section{Files}
    \documentclass{article}

\usepackage[hangul]{kotex}
\usepackage[unicode,bookmarks=true]{hyperref}
\usepackage{fancyvrb}
\usepackage{color}
\usepackage{graphicx}
\usepackage{amsmath}
\usepackage{amsfonts}
\usepackage[top=3cm, left=3cm, right=3cm, bottom=2cm]{geometry}
\usepackage{lipsum}
\begin{document}
    File은 풀이 파일을 제외한 모든 파일들이 보관되는 곳입니다.
    \begin{itemize}
        \item Resource Files : Polygon의 기본 파일과 추가적인 헤더 파일
        \item Source Files : Checker, Validator 등의 파일
        \item Attachment Files : 첨부사진 등 기타 파일
    \end{itemize}
\end{document}

    \section{testlib.h}
    \documentclass{article}

\usepackage[hangul]{kotex}
\usepackage[unicode,bookmarks=true]{hyperref}
\usepackage{fancyvrb}
\usepackage{color}
\usepackage{graphicx}
\usepackage{amsmath}
\usepackage{amsfonts}
\usepackage[top=3cm, left=3cm, right=3cm, bottom=2cm]{geometry}
\usepackage{lipsum}
\begin{document}
    \verb|testlib.h|는 Polygon에서 사용되는 채점용 C++ 헤더입니다. {\color{blue}\href{https://github.com/MikeMirzayanov/testlib}{이 사이트}}에서 다운받을 수 있으며, Polygon의 Checker와 Validator를 짜기 위해서는 필수적으로 알아야 합니다.
\end{document}

    \section{Tests}
    \documentclass{article}

\usepackage[hangul]{kotex}
\usepackage[unicode,bookmarks=true]{hyperref}
\usepackage{fancyvrb}
\usepackage{color}
\usepackage{graphicx}
\usepackage{amsmath}
\usepackage{amsfonts}
\usepackage[top=3cm, left=3cm, right=3cm, bottom=2cm]{geometry}
\usepackage{lipsum}
\begin{document}
    Test에는 test case들이 저장됩니다.
    \subsection{Test 만들기}
    \subsubsection{직접 추가}
    외부에서 만든 test case 추가는 Add Tests를 통해서 진행됩니다. 
    \begin{itemize}
        \item 직접 추가하기 : Data에 Ctrl CV하면 됩니다.
        \item From the archive : test case만 있는 zip 파일을 제출하면 됩니다.
        \item From the files : File을 올리면 됩니다.
    \end{itemize}
    다만 몇 개의 예제를 제외하고는 manual한 방법은 권장되지 않습니다.
    \subsubsection{Generator 작성 (Polygon)}
    \verb|testlib.h|를 사용해서 Generator를 만들 수 있습니다. 
    \begin{itemize}
        \item \verb|void registerGen(int argc, char* argv[], int randomGeneratorVersion)|\\
        이 파일이 Generator라는 것을 선언하는 함수입니다. \verb|argc|와 \verb|argv|는 main의 그것과 같습니다.
        \item \verb|double next()|\\
        0 이상 1 미만 무작위 실수를 반환합니다.
        \item \verb|T next(T n)|\\
        0 이상 \verb|n| 미만의 \verb|T|형 값을 반환합니다.
        \item \verb|T next(T from, T to)|\\
        \verb|from| 이상 \verb|to| 이하의 \verb|T|형 값을 반환합니다.
        \item \verb|std::string next(std::string ptm)|\\
        \verb|ptm|에 있는 Regex 기반 패턴에 맞춘 무작위 문자열을 반환합니다.
        \item \verb|T wnext(T n, int type)|\\
        0 이상 \verb|n| 미만의 T형 수 중, \verb|type|이 양수일 때는 큰 쪽으로, 음수일 때는 작은 쪽으로 가중치가 부여됩니다.
        \item \verb|void shuffle(_RandomAccessIter __first, _RandomAccessIter __last)|\\
        \verb|algorithm| 헤더의 \verb|std::random_shuffle| 대신 사용해야 하는 함수입니다.
        \item \verb|typename Container::value_type any(const Container& c)|\\
        \verb|c| 안에 있는 원소 중 하나가 무작위로 반환됩니다.
    \end{itemize}
    유의해야 할 점은 다음과 같습니다.
    \begin{itemize}
        \item[-] 모든 난수 생성 함수는 \verb|testlib.h|에 전역으로 선언되어 있는 \verb|rnd|을 이용해 호출해야 합니다.
        \item[-] \verb|rand()|나 \verb|srand()|는 사용할 수 없습니다(Polygon이 내부적으로 seeding을 진행합니다)
        \item[-] 인자는 \verb|stdin|대신 \verb|argc|와 \verb|argv|로 전달받습니다.
        \item[-] \verb|testlib.h|의 \verb|generator| 폴더 안에 예시가 동봉되어 있습니다.
    \end{itemize}
    이렇게 만든 Generator는 Files를 통해 업로드 가능하며, test를 생성할 때 Manual 옵션 대신 Script 옵션을 통해 넣을 수도 있습니다.
    \subsection{Generator 사용 (FreeMarker)}
    FreeMarker 기반의 스크립트를 이용하면 여러 개의 test를 Generator를 통해 생성할 수 있습니다. 다음은 예시입니다.
    \begin{verbatim}
    <#assign i = 1/>
    <#list 1..100 as i>
    randomGen 50000 50000 ${i} > $
    </#list>\end{verbatim}
    이 스크립트의 원리는 다음과 같습니다.
    \begin{itemize}
        \item 변수 \verb|i|를 첫 줄에서 할당합니다
        \item 둘째 줄에서 for loop을 돌립니다.
        \item \verb|randomGen|이라는 이름을 가진 generator에 인자 \verb|50000 50000 i| 를 넣어, 그 출력을 아직 만들어지지 않은 가장 낮은 번호의 test(\verb|$|)에 저장합니다.
        \begin{itemize}
            \item \verb|50000 50000|은 (이 경우) generator에게 전달하는 변수입니다.
            \item \verb|${i}|를 넣지 않으면 Polygon이 중복된 test case라고 판단해, 실행이 되지 않습니다.
            \item  또, 추가적으로 들어가는 \verb|${i}|는 seed의 역할도 합니다.
        \end{itemize}
    \end{itemize}
    for문은 중첩될 수도 있으며, \verb|<#assign num = num + 1/>| 등의 문구로 변수의 값을 조절할 수 있습니다.
\end{document}

    \section{Validator}
    \documentclass{article}

\usepackage[hangul]{kotex}
\usepackage[unicode,bookmarks=true]{hyperref}
\usepackage{fancyvrb}
\usepackage{color}
\usepackage{graphicx}
\usepackage{amsmath}
\usepackage{amsfonts}
\usepackage[top=3cm, left=3cm, right=3cm, bottom=2cm]{geometry}
\usepackage{lipsum}
\begin{document}
    Validator는 test case들이 문제의 조건에 부합하는지 검증하는 프로그램입니다.
    \subsection{Validator 작성}
    \verb|testlib.h|를 이용해서 작성할 수 있으며, 입력과 관련된 함수는 \verb|inf|에서 호출해야 합니다. \verb|inf|는 \verb|testlib.h|에 전역으로 선언되어 있는 \verb|InStream|이며, test의 모든 내용이 저장되어 있습니다. 편의상 \verb|inf|를 \textit{포인터}로 지칭하겠습니다.
    \begin{itemize}
        \item \verb|void registerValidation(int argc, char* argv[])|\\
        이 파일이 Validator라는 것을 선언하는 함수입니다. \verb|argc|와 \verb|argv|는 main의 그것과 같습니다.
        \item \verb|void skipBlanks()|\\
        whitespace가 아닌 문자나 EOF를 만날 때까지 포인터를 이동합니다.
        \item \verb|char readChar()|\\
        읽은 문자를 반환하며 포인터를 한 칸 앞으로 움직입니다.
        \item \verb|char readChar(char c)|\\
        \verb|readChar|와 같지만, 읽은 문자가 \verb|c|랑 다르면 에러를 일으킵니다.
        \item \verb|char readSpace()|\\
        \verb|readChar(' ')|와 동일합니다.
        \item \verb|void unreadChar(char c)|\\
        포인터가 가리키고 있는 곳에 \verb|c|를 넣고 포인터를 한 칸 뒤로 옮깁니다. 포인터가 맨 앞에 있을 때 호출하면 에러를 일으킵니다.
        \item \verb|std::string readToken()|, \verb|std::string readWord()|\\
        whitespace아 아닌 문자나 EOF를 읽을 때까지 읽은 문자를 \verb|std::string| 형으로 반환합니다.
        \item \verb|std::string readToken(const std::string& ptrn, const std::string& variableName = "")|\\
        기본적으로는 \verb|readToken|이랑 동일하나 읽은 데이터가 \verb|ptrn|과 일치해야 하고, 일치하지 않을 때 \verb|variableName|을 포함한 에러 메시지가 출력됩니다.
        \item \verb|long long readLong()|\\
        \verb|long long|형의 값을 읽습니다.
        \item \verb|int readInteger(), int readInt()|\\
        \verb|int|형의 값을 읽습니다.
        \item \verb|double readReal(), double readDouble()|\\
        \verb|double|형의 값을 읽습니다.
        \item \verb|long long readLong(long long minv, long long maxv, const std::string& variableName = "")|\\
        \verb|readLong|과 동일하지만 읽은 값이 \verb|minv| 이상 \verb|maxv| 이하여야 하며, 그렇지 않을 경우 \verb|variableName|을 포함한 에러 메시지가 출력됩니다.
        \begin{itemize}
            \item 형만 다른 \verb|readInt|, \verb|readDouble|도 위 꼴로 존재하며, 추가적으로 실수 정밀도까지 다루는 \verb|readStrictDouble| 함수도 존재합니다.
        \end{itemize}
        \item \verb|std::string readString()|, \verb|std::string readLine()|
       줄바꿈(EOLN)이나 EOF를 읽을 때까지 읽은 문자를 \verb|std::string| 형으로 반환합니다.
       \begin{itemize}
           \item 역시 \verb|ptrn|과 \verb|variableName|이 있는 함수도 있으며, 이 쪽이 권장됩니다.
       \end{itemize}
       \item \verb|void readEoln()|\\
       줄바꿈 문자를 읽습니다.
       \item \verb|void readEof()|\\
       EOF를 읽습니다.
       \item \verb|inline void ensuref(bool cond, const char* format, ...)|\\
       \verb|assert(cond)| 구문이랑 동일하며, \verb|format| 이후는 에러 메시지입니다.
    \end{itemize}
    유의해야 할 점은 다음과 같습니다.
    \begin{itemize}
        \item 각 함수들은 반드시 묘사된 의도로만 사용되어야 하며, 그렇지 않을 경우 에러가 날 수 있습니다(예시 : 파일을 아직 다 안 읽었는데 \verb|readEof| 호출).
        \item \verb|testlib.h|의 \verb|validator| 폴더 안에 예시가 동봉되어 있습니다.
    \end{itemize}
    \subsection{Validator 검증}
    만든 Validator를 검증하기 위한 테스트 데이터를 Validator - Add Test를 통해 만들 수 있습니다.
    \begin{itemize}
        \item 각 test에 대한 Input과 그에 따른 Verdict(결과)를 입력할 수 있습니다.
        \item Multiple Tests 옵션을 활성화하면 \verb|===|을 줄의 경계로 여러 test를 넣을 수 있습니다.
        \begin{itemize}
            \item Validator의 test는 generator나 script 등으로 넣을 수 없으므로 이 쪽이 권장됩니다.
            \item 단, 메가바이트 단위의 텍스트를 Ctrl CV하는 것은 브라우저의 렉을 유발할 수 있습니다.
        \end{itemize}
        \item 실제 위 test로 검증하는 것은 나중에 Main Solution을 설정하고 진행할 수 있습니다.
    \end{itemize}
\end{document}

    \section{Solution Files}
    \documentclass{article}

\usepackage[hangul]{kotex}
\usepackage[unicode,bookmarks=true]{hyperref}
\usepackage{fancyvrb}
\usepackage{color}
\usepackage{graphicx}
\usepackage{amsmath}
\usepackage{amsfonts}
\usepackage[top=3cm, left=3cm, right=3cm, bottom=2cm]{geometry}
\usepackage{lipsum}

\begin{document}
    Solution Files는 풀이를 올리는 곳입니다. Add Solution을 통해 업로드할 수 있습니다. 각 풀이는 다음과 같이 분류됩니다.
    \begin{itemize}
        \item \textbf{Main correct solution} : 정해이며, 유일합니다. 이 코드의 출력이 정답으로 간주됩니다.
        \item \textbf{Correct} : 정해와 일치하거나 인정받는 풀이입니다.
        \item \textbf{Incorrect} : 여러 가지로 틀린 풀이입니다. 모든 Verdict(결과)가 허용됩니다.
        \item \textbf{Time limit exceeded} : 시간제한을 초과한 풀이입니다.
        \item \textbf{Wrong answer} : 출력이 올바르지 않은 풀이입니다.
        \item \textbf{Memory limit exceeded} : 메모리 제한을 넘은 풀이입니다.
        \item \textbf{Failed} : 실행 중 오류가 나는 풀이입니다.
    \end{itemize}
    이 중 Incorrect를 제외하고는 각 풀이는 Correct나 해당하는 오류만 허용됩니다. 즉 Wrong answer인 풀이가 Time limit exceeded나 Failed가 나오면 안 됩니다.
\end{document}

    %\section{Checker}
    %\documentclass{article}

\usepackage[hangul]{kotex}
\usepackage[unicode,bookmarks=true]{hyperref}
\usepackage{fancyvrb}
\usepackage{color}
\usepackage{graphicx}
\usepackage{amsmath}
\usepackage{amsfonts}
\usepackage[top=3cm, left=3cm, right=3cm, bottom=2cm]{geometry}
\usepackage{lipsum}

\begin{document}
    Checker는 풀이가 맞는지 틀린지 판별하는 프로그램입니다.
    \subsection{Checker의 구성}
    Validator에서 \verb|inf|가 있었던 것처럼, 제출한 답안의 풀이는 \verb|ouf|에, 모범 답안은 \verb|ans|에 저장되어 있습니다. Validator에서 사용했던 함수들로 읽어낼 수 있고, 추가적으로 다음 함수들을 사용합니다.
    \begin{itemize}
        \item \verb|void registerTestlibCmd(int argc, char* argv[])|\\
        이 파일이 Checker라는 것을 선언하는 함수입니다. \verb|argc|와 \verb|argv|는 main의 그것과 같습니다.
        \item \verb|void quit(TResult verdict, string message)|\\
        \verb|void quit(TResult verdict, const char* message)|\\
        \verb|void quitf(TResult verdict, const char* message, ...)|\\
        Checker가 내린 판단을 \verb|verdict|에 저장하고 \verb|message|를 출력합니다. \verb|verdict|에 들어갈 수 있는 값은 다음과 같습니다.
        \begin{itemize}
            \item \verb|_ok| : OK, 정답
            \item \verb|_wa| : WA, 오답
            \item \verb|_pc| : Partial score (부분점수)
            \item \verb|_fail| : Special Judge에서 모범 답안보다 훌륭한 값이 나왔을 때 사용
        \end{itemize}
    \end{itemize}
    Checker는 3개의 수를 출력하라고 했는데 4개의 수를 출력하는 코드나 범위에 어긋난 수를 출력하는 코드 등 괴상한 상황도 처리해야 하며, Special Judge의 경우 답이 바른지도 처리해야 합니다. 치밀하게 구성되어야 하며, 길라잡이로 \verb|testlib.h|의 \verb|checker| 폴더에 예시가 동봉되어 있습니다.
    \subsection{Checker 검증}
    Checker를 검증하는 방법은 Validator를 검증하는 방식과 거의 동일합니다.
    \subsection{Checker까지 만들었으면}
    여기까지 진행했으면 Tests / Validator Test / Checker Test를 돌리는 것이 가능합니다.
\end{document}

\end{document}
