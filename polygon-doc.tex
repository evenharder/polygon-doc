\documentclass{article}

\usepackage[hangul]{kotex}
\usepackage[hidelinks,unicode,bookmarks=true]{hyperref}
\hypersetup{bookmarksdepth=3}
\usepackage{fancyvrb}
\usepackage{color}
\usepackage{graphicx}
\usepackage{amsmath}
\usepackage{amsfonts}
\usepackage{amsthm}
\usepackage{textcomp}
\usepackage[top=3cm, left=3cm, right=3cm, bottom=2cm]{geometry}
\usepackage{lipsum}
\usepackage{titling}
\usepackage{colortbl}
\usepackage{standalone}

\setlength{\parindent}{0pt}
\setlength{\droptitle}{-3cm}
\setcounter{tocdepth}{1}

\title{간단하게 배워보는 Polygon}
\author{작성자 : \texttt{@evenharder}}
\date{최종날짜 : \today}
\begin{document}
    \maketitle
    이 문서는 Polygon (\url{https://polygon.codeforces.com/}) 의 간략한 소개를 다룹니다.
    \tableofcontents
    \newpage
    \section{Introduction}
    \documentclass{article}

\usepackage[hangul]{kotex}
\usepackage[unicode,bookmarks=true]{hyperref}
\usepackage{fancyvrb}
\usepackage{color}
\usepackage{graphicx}
\usepackage{amsmath}
\usepackage{amsfonts}
\usepackage[top=3cm, left=3cm, right=3cm, bottom=2cm]{geometry}
\usepackage{lipsum}
\begin{document}
    \begin{quotation}
        The mission of Polygon is to provide platform for creation of programming contest problems. Polygon supports the whole development cycle: 
        problem statement writing,
        test data preparing (generators supported),
        model solutions (including correct and wittingly incorrect),
        judging, and
        automatic validation.
        \begin{flushright}
            \textit{Polygon Index Page}
        \end{flushright}
    \end{quotation}
    메인 화면에 적혀있는 Polygon의 목표로, 프로그래밍 대회 문제 제작을 위한 플랫폼을 제공하는 것입니다.
\end{document}

    \section{Registration}
    \documentclass{article}

\usepackage[hangul]{kotex}
\usepackage[unicode,bookmarks=true]{hyperref}
\usepackage{fancyvrb}
\usepackage{color}
\usepackage{graphicx}
\usepackage{amsmath}
\usepackage{amsfonts}
\usepackage[top=3cm, left=3cm, right=3cm, bottom=2cm]{geometry}
\usepackage{lipsum}
\begin{document}
    메인 화면 우측 상단의 \textit{Register}로 등록 가능합니다.
\end{document}

    \section{Creating Problem}
    \documentclass{article}

\usepackage[hangul]{kotex}
\usepackage[unicode,bookmarks=true]{hyperref}
\usepackage{fancyvrb}
\usepackage{color}
\usepackage{graphicx}
\usepackage{amsmath}
\usepackage{amsfonts}
\usepackage[top=3cm, left=3cm, right=3cm, bottom=2cm]{geometry}
\usepackage{lipsum}
\begin{document}
    로그인을 하고, 상단의 \textit{New Problem}을 통해 문제의 코드네임(예시 : \verb|books|, \verb|hanoi|, \verb|holiday|)을 설정할 수 있습니다.
    \subsection*{Commit Changes}
    Git의 commit 기능처럼, Polygon 서버에 지금까지의 과정을 저장하고 싶을 때 우측 하단의 \textit{Commit Changes}를 누르면 Commit 과정을 진행할 수 있습니다.
\end{document}

    \section{General Info}
    \documentclass{article}

\usepackage[hangul]{kotex}
\usepackage[unicode,bookmarks=true]{hyperref}
\usepackage{fancyvrb}
\usepackage{color}
\usepackage{graphicx}
\usepackage{amsmath}
\usepackage{amsfonts}
\usepackage[top=3cm, left=3cm, right=3cm, bottom=2cm]{geometry}
\usepackage{lipsum}
\begin{document}
    General Info는 문제 자체의 특징을 다룹니다.
    \subsection{General Info의 구성}
    \begin{itemize}
        \item Input file : 입력 방식으로, \verb|stdin|을 권장합니다.
        \item Output file : 출력 방식으로, \verb|stdout|을 권장합니다.
        \item Time limit : 시간 제한(250ms - 15000ms)을 설정합니다.
        \item Memory limit : 메모리 제한(4MB - 1024MB)을 설정합니다.
        \item Interactive : 문제가 interactive한지의 여부를 의미합니다.
        \item Well-formed : 체크하면 whitespace나 control character를 Polygon이 알아서 처리해줍니다.
        \begin{itemize}
            \item 이 경우 자동으로 모든 test case가 whitespace 관련 제약조건을 만족하게 변경됩니다.
        \end{itemize}
    \end{itemize}
\end{document}

    \section{Statement}
    \documentclass{article}

\usepackage[hangul]{kotex}
\usepackage[unicode,bookmarks=true]{hyperref}
\usepackage{fancyvrb}
\usepackage{color}
\usepackage{graphicx}
\usepackage{amsmath}
\usepackage{amsfonts}
\usepackage[top=3cm, left=3cm, right=3cm, bottom=2cm]{geometry}
\usepackage{lipsum}
\begin{document}
    Polygon에서 문제 설명은 영어나 러시아어로 쓰는 게 권장됩니다.
    \subsection{Statement의 구성}
    \begin{itemize}
        \item Encoding : UTF-8 그대로가 권장됩니다.
        \item Name : 문제의 제목입니다.
        \item Legend : 배경 지식 및 문제 상황 설명입니다.
        \item Input Format : 입력 양식입니다.
        \item Output Format : 출력 양식입니다.
        \item Note : 예제가 나오는 이유 등을 적는 곳입니다.
        \item Tutorial : 문제 페이지에 보이진 않지만, 풀이를 적는 곳입니다.
    \end{itemize}
    여기까지의 내용은 페이지 하단의 Save를 눌러야 저장됩니다.
    \subsection{Statement 속 TeX 사용}
    Polygon의 Statement는 TeX을 이용해 컴파일됩니다.
    \subsubsection{TeX이 친숙하지 않을 때}
    \begin{itemize}
        \item TeX에선 연속된 whitespace는 하나로 처리됩니다.
        \item 줄바꿈은 \verb|\\|입니다.
        \item 수식은 \verb|$...$|나 \verb|$$...$$|의 꼴로 이루어집니다.
        \item 수식에서 윗첨자는 \verb|^|, 아랫첨자는 \verb|_|입니다.
        \item 그림도 집어넣을 수 있지만, Polygon 특성상 까다롭습니다.
        \item 특수 기호 사용법은 {\color{blue}\href{http://detexify.kirelabs.org/classify.html}{이 사이트}}에서 검색할 수 있습니다.
    \end{itemize}
    \subsubsection{TeX이 친숙할 때}
    \begin{itemize}
        \item 각 항목을 편집할 때 \verb|\usepackage| 구문은 필요없습니다.
        \item 복잡하거나 특수한 TeX 구문은 컴파일되지 않을 수 있습니다.
    \end{itemize}
\end{document}

\end{document}
