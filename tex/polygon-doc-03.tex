\documentclass{article}

\usepackage[hangul]{kotex}
\usepackage[unicode,bookmarks=true]{hyperref}
\usepackage{fancyvrb}
\usepackage{color}
\usepackage{graphicx}
\usepackage{amsmath}
\usepackage{amsfonts}
\usepackage[top=3cm, left=3cm, right=3cm, bottom=2cm]{geometry}
\usepackage{lipsum}
\begin{document}
    로그인을 하고, 상단의 \textit{New Problem}을 통해 문제의 코드네임(예시 : \verb|books|, \verb|hanoi|, \verb|holiday|)을 설정할 수 있습니다.
    \subsection*{Commit Changes}
    Git의 commit 기능처럼, Polygon 서버에 지금까지의 과정을 저장하고 싶을 때 우측 하단의 \textit{Commit Changes}를 누르면 Commit 과정을 진행할 수 있습니다.
\end{document}
