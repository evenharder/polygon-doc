\documentclass{article}

\usepackage[hangul]{kotex}
\usepackage[unicode,bookmarks=true]{hyperref}
\usepackage{fancyvrb}
\usepackage{color}
\usepackage{graphicx}
\usepackage{amsmath}
\usepackage{amsfonts}
\usepackage[top=3cm, left=3cm, right=3cm, bottom=2cm]{geometry}
\usepackage{lipsum}
\begin{document}
    General Info는 문제 자체의 특징을 다룹니다.
    \subsection{General Info의 구성}
    \begin{itemize}
        \item Input file : 입력 방식으로, \verb|stdin|을 권장합니다.
        \item Output file : 출력 방식으로, \verb|stdout|을 권장합니다.
        \item Time limit : 시간 제한(250ms - 15000ms)을 설정합니다.
        \item Memory limit : 메모리 제한(4MB - 1024MB)을 설정합니다.
        \item Interactive : 문제가 interactive한지의 여부를 의미합니다.
        \item Well-formed : 체크하면 whitespace나 control character를 Polygon이 알아서 처리해줍니다.
        \begin{itemize}
            \item 이 경우 자동으로 모든 test case가 표시되는 제약조건을 만족하게 변경됩니다.
        \end{itemize}
    \end{itemize}
\end{document}
