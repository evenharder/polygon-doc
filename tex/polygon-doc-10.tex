\documentclass{article}

\usepackage[hangul]{kotex}
\usepackage[unicode,bookmarks=true]{hyperref}
\usepackage{fancyvrb}
\usepackage{color}
\usepackage{graphicx}
\usepackage{amsmath}
\usepackage{amsfonts}
\usepackage[top=3cm, left=3cm, right=3cm, bottom=2cm]{geometry}
\usepackage{lipsum}

\begin{document}
    Solution Files는 풀이를 올리는 곳입니다. Add Solution을 통해 업로드할 수 있습니다. 각 풀이는 다음과 같이 분류됩니다.
    \begin{itemize}
        \item \textbf{Main correct solution} : 모범 답안이며, 유일합니다. 이 코드의 출력이 정답으로 간주됩니다.
        \item \textbf{Correct} : 모범 답안과 일치하거나 인정받는 풀이입니다.
        \item \textbf{Incorrect} : 여러 가지로 틀린 풀이입니다. 모든 Verdict(결과)가 허용됩니다.
        \item \textbf{Time limit exceeded} : 시간제한을 초과한 풀이입니다.
        \item \textbf{Wrong answer} : 출력이 올바르지 않은 풀이입니다.
        \item \textbf{Memory limit exceeded} : 메모리 제한을 넘은 풀이입니다.
        \item \textbf{Failed} : 실행 중 오류가 나는 풀이입니다.
    \end{itemize}
    이 중 Incorrect를 제외하고는 각 풀이는 Correct나 해당하는 오류만 허용됩니다. 즉 Wrong answer인 풀이가 Time limit exceeded나 Failed가 나오면 안 됩니다.
\end{document}
