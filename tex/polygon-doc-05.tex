\documentclass{article}

\usepackage[hangul]{kotex}
\usepackage[unicode,bookmarks=true]{hyperref}
\usepackage{fancyvrb}
\usepackage{color}
\usepackage{graphicx}
\usepackage{amsmath}
\usepackage{amsfonts}
\usepackage[top=3cm, left=3cm, right=3cm, bottom=2cm]{geometry}
\usepackage{lipsum}
\begin{document}
    Polygon에서 문제 설명은 영어나 러시아어로 쓰는 게 권장됩니다.
    \subsection{Statement의 구성}
    \begin{itemize}
        \item Encoding : UTF-8 그대로가 권장됩니다.
        \item Name : 문제의 제목입니다.
        \item Legend : 배경 지식 및 문제 상황 설명입니다.
        \item Input Format : 입력 양식입니다.
        \item Output Format : 출력 양식입니다.
        \item Note : 예제가 나오는 이유 등을 적는 곳입니다.
        \item Tutorial : 문제 페이지에 보이진 않지만, 풀이를 적는 곳입니다.
    \end{itemize}
    여기까지의 내용은 페이지 하단의 Save를 눌러야 저장됩니다.
    \subsection{Statement 속 TeX 사용}
    Polygon의 Statement는 TeX을 이용해 컴파일됩니다.
    \subsubsection{TeX이 친숙하지 않을 때}
    \begin{itemize}
        \item TeX에선 연속된 whitespace는 하나로 처리됩니다.
        \item 줄바꿈은 \verb|\\|입니다.
        \item 수식은 \verb|$...$|나 \verb|$$...$$|의 꼴로 이루어집니다.
        \item 수식에서 윗첨자는 \verb|^|, 아랫첨자는 \verb|_|입니다.
        \item 그림도 집어넣을 수 있지만, Polygon 특성상 까다롭습니다.
        \item 특수 기호 사용법은 {\color{blue}\href{http://detexify.kirelabs.org/classify.html}{이 사이트}}에서 검색할 수 있습니다.
    \end{itemize}
    \subsubsection{TeX이 친숙할 때}
    \begin{itemize}
        \item 각 항목을 편집할 때 \verb|\usepackage| 구문은 필요없습니다.
        \item 복잡하거나 특수한 TeX 구문은 컴파일되지 않을 수 있습니다.
    \end{itemize}
\end{document}
